\title{Multislice Diffraction Theory}

\author{Erik Bigwood}

\date{7/23/17}

\documentclass{article}
\usepackage{amsmath}
\usepackage{amssymb}
\usepackage{amsthm}
\usepackage{csvsimple}
\usepackage{tabularx}
\usepackage{mathtools}
\usepackage{graphicx}
\usepackage{pgfplots}
\usepackage{tikz}
\usepackage{gensymb}
\usepackage{hyperref}
\pgfplotsset{compat=1.5}
\usepackage{multicol}
\usepackage{wrapfig}
\usepackage{float}
\usepackage{hyperref}
\hypersetup{
    colorlinks=true,
    linkcolor=black,
    filecolor=magenta,      
    urlcolor=blue,
}
 
\urlstyle{same}

\graphicspath{Images/}
\DeclarePairedDelimiter\ceil{\lceil}{\rceil}
\DeclarePairedDelimiter\floor{\lfloor}{\rfloor}

\begin{document}
	\maketitle
	\tableofcontents
	\newpage
	\pagenumbering{Roman}
	\section{Functions}
		\subsection{lambda\_from\_eV}
			This function takes an energy in eV and returns the 	relativistic electron wavelength. The equation is shown below:
			\begin{equation}
				\lambda = \sqrt{\frac{h^2c^2}{E^2-m_0^2 c^2}}
			\end{equation}
	
		\subsection{rotate\_vec\_array}
			Iterates over each vector in the n x 3 array $\Lambda$ and rotates them around $\hat{x}$, $\hat{y}$, then $\hat{z}$ by $tx$, $t$y, and $tz$ (rad) respectively.
			That is, 
			\begin{equation}
				\Bigg(\forall v\in \Lambda\Bigg)\Bigg(v\to R_z(\theta_z)\cdot R_y(\theta_y)\cdot R_x(\theta_x) \cdot v = \mathcal{R}v\Bigg)
			\end{equation}
	
		\paragraph{rotation\_mat}
			Returns a matrix corresponding to a rotation around $\hat{x}$, $\hat{y}$, then $\hat{z}$ by $tx$, $t$y, and $tz$ (rad) respectively.\footnote{See  \href{https://en.wikipedia.org/wiki/Rotation\_matrix}{rotation matrices}.}
				
			That is,
				\begin{equation}
					\mathcal{R}=R_z(\theta_z)\cdot R_y(\theta_y)\cdot R_x(\theta_x)
				\end{equation}
			
			\subparagraph{rotation\_mat\_x}
				Returns the rotation matrix around $\hat{x}$ by $\theta_x$, $R_x(\theta_x)$.
				
			\subparagraph{rotation\_mat\_y}
				Returns the rotation matrix around $\hat{y}$ by $\theta_y$, $R_y(\theta_y)$.
			
			\subparagraph{rotation\_mat\_z}
				Returns the rotation matrix around $\hat{z}$ by $\theta_z$, $R_z(\theta_z)$.
\end{document}